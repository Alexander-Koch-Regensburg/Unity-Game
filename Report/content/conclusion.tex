\section{Zusammenfassung und Ausblick}\label{sec:conclusion}

In der vorliegenden Arbeit wurde die Entwicklung eines Spiels auf Basis der Unity-Spiele-Engine vorgestellt. Das Ziel war es, den Spieler aus der Vogelperspektive durch ein zweidimensionales Level steuern zu können. Hierbei kann der Spieler mit verschiedenen Waffentypen interagieren um sich gegen unterschiedliche Typen von Gegnern zu behaupten. Ein Level gilt als gewonnen sobald der Spieler einen spezifizierten Endbereich erreicht oder alle Gegner eliminiert hat. 

Hierzu wurde zunächst in Kapitel \ref{sec:introduction} die grundlegende Spielidee ausführlich erläutert und die geplanten Spielinhalte charakterisiert. Anschließend wurden die wichtigsten technischen Grundlagen von Unity vorgestellt. Im Anschluss daran wurde in Kapitel \ref{sec:levelArchitecture} der grundsätzliche Aufbau eines Levels sowie aller darin enthaltenen Elemente beschrieben. Nachstehend wurde der Grundaufbau von Charakteren im Spiel sowie die Steuerung des Spielers und die Verwendung der verschiedenen Waffentypen in Kapitel \ref{sec:charactersAndItems} erläutert. Da das Design, das Verhalten und der von den Gegnern genutzte Wegfindungsalgorithmus zentrale und umfangreiche Komponenten des Projekts darstellen, wurden diese in einem eigenen Kapitel \ref{sec:enemiesAndAI} ausführlich dargestellt. Für das Abspielen von Geräuschen wurde ein Audiosystem entwickelt, dessen Aufbau und Funktionalität in Kapitel \ref{audio} wiedergegeben wird. Des Weiteren ist es möglich, einen aktuellen Spielstand zu sichern und zu einem späteren Zeitpunkt wiederherzustellen, wie in Kapitel \ref{sec:designSerialization} beschrieben ist. Die Planung und Umsetzung einer Schnittstelle zur Steuerung der Charaktere von außen ist in Kapitel \ref{sec:aiInterface} ausgeführt. In Kapitel \ref{leveleditor} werden verschiedene Ansätze zur Gestaltung eines Level-Editors beschrieben und die Implementierung einer Variante dargestellt. Abschließend wird in Kapitel \ref{sec:testing} auf das Testen von Software in Verbindung mit der Unity-Spiele-Engine eingegangen und in diesem Zuge Schwierigkeiten und Lösungsmöglichkeiten aufgezeigt. 

Insgesamt konnte die Spielidee soweit realisiert werden und die geplanten Kerninhalte sind in die Software integriert. Es ist jedoch anzumerken, dass einige in der frühen Designphase angedachten Spielinhalte aus zeitlichen Gründen nicht in die resultierende Software mit aufgenommen werden konnten. 

Dazu zählen unter anderem weitere Level-Elemente, wie beispielsweise Fenster, durch die geschossen, aber nicht gelaufen werden kann. Außerdem bestand eine ursprüngliche Idee darin, Lüftungsschächte an Wänden hinzuzufügen, durch die sich nur der Spieler leise zwischen Räumen fortbewegen kann. Die bereits bestehenden Türen könnten in Zukunft außerdem noch um optionale Konsolen erweitert werden, mit denen sich diese versperren lassen können. Die grundlegende Logik hierfür ist bereits vorhanden. Des Weiteren wären aus Sicht der Gegner KI noch Überwachungskameras interessant, die bei Entdeckung des Spielers alle umliegenden Gegner alarmieren. Mit den genannten Level-Elementen könnten die Spiellevel noch deutlich abwechslungsreicher und anspruchsvoller gestaltet werden, wobei die Möglichkeiten für Erweiterungen noch deutlich über dies hinausgehen.

Auch im Bezug auf Waffen und benutzbare Gegenstände gibt es für die Zukunft viele Expansionsmöglichkeiten. Eine Idee wäre zum Beispiel, Granaten für den Spieler einzuführen, die Flächenschaden verursachen würden oder Lebens-Items, die einen Teil der Lebensenergie des Spielers wiederherstellen.

Neben den direkt sichtbaren Erweiterungen des Levels oder der Gegenstände bestehen auch in der Hintergrundlogik des Spiels Möglichkeiten zur weiteren Verbesserung. Im Bezug auf den Pfadfindealgorithmus wäre es interessant alternative Heuristiken, die zum Beispiel raumbasiert anhand des Levels berechnet werden, auf ihre Performanz hin zu testen. Ebenso könnten auch gänzlich andere Pfadfindealgorithmen eingesetzt werden.

Am meisten Potenzial bietet aber die Schnittstelle für die Gegner und den Spieler. Diese wurde explizit für die zukünftige Weiterarbeit am Projekt implementiert und ermöglicht beispielsweise durch maschinelles Lernen die Gegner noch deutlich intelligenter zu machen, und so auch Erfahrung in diesem Gebiet im Kontext von Videospielen zu sammeln.

Abgesehen von den eher technischen Änderungen ist vor allem die Ästhetik des Spiels noch ausbaufähig. So würden detailliertere Texturen für den Spieler und die Gegner als auch das Anwenden eines einheitlicheren Grafikstils die Software noch deutlich mehr nach einem Spiel aussehen lassen. Das Hinzufügen einer Storykampagne könnte ebenso den Spielspaß deutlich verbessern und als Einführung in das Spiel dienen, auch wenn mit dem Level-Editor bereits eigene Spiellevel erstellt werden können. Ein interaktives Tutorial könnte alternativ auch dem Nutzer beim Spieleinstieg behilflich sein.

Summa summarum ist es innerhalb des Projekts gelungen, ein funktionstüchtiges Grundspiel zu realisieren. Dennoch sollten die vielen offenen Möglichkeiten zur Erweiterung in Zukunft genutzt werden, um ein insgesamt vollständigeres Produkt zu erhalten, vor allem, weil bei der Implementierung von Anfang an auf die Erweiterbarkeit der Software geachtet wurde. Einige der in der Konzeptphase ursprünglich erdachten Spielinhalte hätten zwar noch in das Spiel integriert werden können, beispielsweise durch die Verwendung von Unity spezifischen vorgefertigten Lösungen zur Wegfindung der Gegner, jedoch war es so möglich, einen tieferen Einblick in die darunter verborgene Logik zu erhalten, was zu einer subjektiv besseren Lernerfahrung geführt hat.

%, wenn zum Beispiel zur Wegfindung der Gegner von Unity vorgefertigte Lösungen verwendet worden wären,